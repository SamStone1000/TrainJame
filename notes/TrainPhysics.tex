% Created 2022-08-09 Tue 23:39
% Intended LaTeX compiler: pdflatex
\documentclass[11pt]{article}
\usepackage[utf8]{inputenc}
\usepackage[T1]{fontenc}
\usepackage{graphicx}
\usepackage{longtable}
\usepackage{wrapfig}
\usepackage{rotating}
\usepackage[normalem]{ulem}
\usepackage{amsmath}
\usepackage{amssymb}
\usepackage{capt-of}
\usepackage{hyperref}
\author{Stone}
\date{\today}
\title{}
\hypersetup{
 pdfauthor={Stone},
 pdftitle={},
 pdfkeywords={},
 pdfsubject={},
 pdfcreator={Emacs 28.1 (Org mode 9.5.2)}, 
 pdflang={English}}
\begin{document}

\tableofcontents

Train Physics -\textbf{- mode: org -}-

\section{Starting}
\label{sec:orga95f3ab}
\subsection{Maximum Load}
\label{sec:org806b88b}
\begin{itemize}
\item The maximum force a locomotive can output is equal to the friction
between the wheels and rail
\begin{itemize}
\item In other words the most force a locomotive can output (unless
the engine can't produce more than this) is the locomotive's
weight multiplied by the coeefficent of static friction of the
wheels and rail (\textasciitilde{}.5 for steel-steel)
\item The most force a locomotive can output while stationary is
called \textbf{starting tractive effort}
\item Thus if the train's static friction is greater then the starting
tractive effort of all the locomotive's put together it cannot
move, and trying to push the locomotives further will simply
cause the wheels to spin
\end{itemize}
\end{itemize}
\subsection{Accelerating}
\label{sec:org8a3720a}
\begin{itemize}
\item The amount of power a locomotive can output is inversely
proportional to the speed of the train
\begin{itemize}
\item This means as the train accelerates and picks up speed, it's
rate of accleration drops as the force output by the engine
drops
\begin{itemize}
\item The force for a desiel-electric motor can be approximated as
\texttt{T=(Pn)/V}
\begin{description}
\item[{T}] tractive effort of locomotive in newtons
\item[{P}] power of locomotive in watts
\item[{n}] efficency of locomotive in converting power to force
\item[{V}] speed of locomotive in m/s
\end{description}
\end{itemize}
\item Eventually this means the train can't accelerate past the point
where the engine's tractive effort equals the rolling resistance
of the train + air resistance
\end{itemize}
\end{itemize}
\subsubsection{Math}
\label{sec:org756d1f8}

\section{While Moving}
\label{sec:org80380d1}
\begin{itemize}
\item Once a train reaches the equilibrium point where tractive effort is
equal to resistances, the train will stop accelerating and stay at a
constant speed
\end{itemize}

\begin{equation}
\label{eq:force}
F = m \cdot a
\end{equation}

Force is given by equation \ref{eq:force} above.
\end{document}